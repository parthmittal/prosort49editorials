\section{Problem B}
    \subsection{Statement}
        Given integers $a$, $b$, and $m$, find $a^b \bmod m$

    \subsection{Solution}
        Since $b \leq 10^{18}$, we cannot use naive exponentiation.
        There is a well known trick to do exponentiation in
        $O(\log b)$ multiplications, we present it as is.

        \begin{algorithm}[H]
        \begin{algorithmic}[1]
        \Function{mod-exp}{$a, b, m$}
            \State $R \gets 1, C \gets a \bmod m$
            \While{$b > 0$}
                \If{$b \bmod 2 = 1$}
                    \State $R \gets \left(R \times C\right) \bmod m$
                \EndIf
                \State $C \gets \left(C \times C\right) \bmod m$
                \State $b \gets \floor*{\frac b2}$
            \EndWhile
        \EndFunction
        \end{algorithmic}
        \end{algorithm}

        For a second, it looks like we are done here, look at lines
        5 and 7 of the algorithm.
        Since $R$ and $C$ are bounded only by $m$, they can be up to
        $10^{18}$ before these multiplications are performed.
        So the result of the multiplication does not fit in a 64-bit
        integer type.

        To encourage people to think about how to get around this, we
        disallowed anything but C and C++ for this problem (and hoped
        nobody knew about \verb|__int_128|).

        (hint: note that multiplication is just repeated addition, like
        exponentiation is repeated multiplication)

        Rather cleverly, it turns out we can use the above algorithm as is,
        just replacing the multiply bits with the addition, and setting
        $R$ to 0 (the identity of addition) instead of $1$ (the identity
        of multiplication).

        \begin{algorithm}[H]
        \begin{algorithmic}[1]
        \Function{mod-mul}{$a, b, m$}
            \State $R \gets 0, C \gets a \bmod m$
            \While{$b > 0$}
                \If{$b \bmod 2 = 1$}
                    \State $R \gets \left(R + C\right) \bmod m$
                \EndIf
                \State $C \gets \left(C + C\right) \bmod m$
                \State $b \gets \floor*{\frac b2}$
            \EndWhile
        \EndFunction
        \end{algorithmic}
        \end{algorithm}

        This works fine, since $10^{18} + 10^{18}$ still fits within a 64-bit
        integer type.
        And of course, we replace lines 5 and 7 in the first algorithm with calls
        to \textproc{mod-mul}, for an overall run-time of $O(\log b \times \log m)$
        per test case, and $O(T \times \log b \times \log m)$ per test file.

    \subsection{Comments}
        Interestingly, this was the only semi-original problem in this ProSort
        (yes, we are incredibly creative problem authors),
        in that I learned this trick somewhere (cannot remember where $\ddot\smallfrown$)
        and decided everybody should know it because reasons.